\mySection{Planl\ae g Din Tr\ae ning!}
\begin{multicols}{2}
  \begin{tList}{Den allerbedste træningsplan er den du følger}
  \item Det vigtigste er, at du rent faktisk laver din træning - den
    bedste plan er den, du følger. Hvis du tilføjer en masse seje
    øvelser, som du alligevel kun får lavet engang imellem, kan du
    lige så godt skære ned, så du kan sikre dig at du rent faktisk
    får det gjort.
  \item Hvis træningsplanen er simpel og overkommelig, er der også
    større chance for at du bliver ved med at følge den i lang tid.
  \item  Det er svært at blive stærkere, hvis man hele tiden hopper
    fra plan til plan.
  \end{tList}

\columnbreak

  \begin{tList}{Minimum effective dose aka keep it simple}
  \item Du skal træne så lidt som muligt, så længe du bliver stærkere
    af det. Hvorfor det?

  \item Du bruger mindre energi, så har du mere krudt til klatring
  \item Du bliver ikke overvældet af tusind forskellige øvelser
  \item Det er sjovere at klatre end det er at træne. Mindre tid
    træning, mere tid klatring

  \item Væk fra no pain no gain mentalitet. Du er ikke bodybuilder.
  \item Hvis din træning er simpel og kort, er der større
    sandsynlighed for at du gør det.
  \end{tList}

\end{multicols}
