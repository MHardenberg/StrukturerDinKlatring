\mySection{S\aa dan kunne en tr\ae ningsplan se ud}

\tsection{Plan for 3 tr\ae ningsdage om ugen}
\begin{table}[h!]
  \centering
        \rowcolors{2}{gray!15}{white}
  \begin{tabular}{c| c c c c}
    \textbf{ugedag} & \textbf{aktivitet} & \\ \hline
    Mandag & Varm op & Hangboard & Overh\ae ng (20-30 min.) & Fri klatring / tr\ae ning \\
    Onsdag & Varm op & Fri klatring & Pull ups \& anden tr\ae ning & \\
    Fredag & Varm op & Hangboard & Fri klatring og hyg dig! & \\
  \end{tabular}
\end{table}
  \vspace{1em}
 
Hangboard før klatring fordi man er for træt til det efter klatring. Det er en prioritet at få hangboardtræning af høj kvalitet, og det er en prioritet ikke at være træt og så bare køre på alligevel og rive sin finger over.

Vurder graden af planl\ae gning! Undg{\aa} at gøre klatring kedeligt med en for striks struktur.
At have det sjovt hj\ae lper med at holder jer igang.
\vspace{2em}


\mySection{N\aa r i laver/f\o lger jeres plan}
\begin{multicols}{2}


  \begin{tList}{Skriv det ned!}
  \item Hvis man tager noter til sin træning, er man aldrig i tvivl
    om, hvorvidt det virker eller ej. Bare kig på tallene.
  \item Det bliver meget lettere at vurdere, hvor meget man skal
    løfte fra gang til gang, når man ikke bare gætter ud fra mavefornemmelse.
  \end{tList}

  \begin{tList}{Langsigtet perspektiv}
  \item Det handler ikke om at blive stærk nok til at lukke det næste
    projekt, men at blive stærk år efter år.
  \item Start der hvor du rent faktisk er, og ikke der, hvor du ville
    ønske du var.

  \item Ønsketænkning og selvbedrag har forvoldt mange skader.
  \end{tList}




\end{multicols}
