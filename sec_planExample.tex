\mySection{S\aa dan kunne en tr\ae ningsplan se ud}

\tsection{Plan for 3 tr\ae ningsdage om ugen}
\begin{table}[h!]
  \centering
        \rowcolors{2}{gray!15}{white}
  \begin{tabular}{c| c c c c}
    \textbf{ugedag} & & \\ \hline
    Mandag & Varm op & Hangboard & Overh\ae ng (20-30 min.) & Fri klatring / tr\ae ning \\
    Onsdag & Varm op & Fri klatring & Pull ups \& anden tr\ae ning & \\
    Fredag & Varm op & Hangboard & Fri klatring og hyg dig! & \\
  \end{tabular}
\end{table}
  \vspace{1em}
 
Hangboard før klatring fordi man er for træt til det efter klatring. Det er en prioritet at få hangboardtræning af høj kvalitet, og det er en prioritet ikke at være træt og så bare køre på alligevel og rive sin finger over.

Vurder graden af planl\ae gning! Undg{\aa} at gøre klatring kedeligt med en for striks struktur.
At det er sjovt hj\ae lper med at holder jer igang.
