\begin{multicols}{2}
\mySection{Planl\ae g Din Tr\ae ning!}
\begin{tList}{Den allerbedste træningsplan er den du følger}
\item Det vigtigste er, at du rent faktisk laver din træning - den bedste plan er den, du følger. Hvis du tilføjer en masse seje øvelser, som du alligevel kun får lavet engang imellem, kan du lige så godt skære ned, så du kan sikre dig at du rent faktisk får det gjort.
\item Hvis træningsplanen er simpel og overkommelig, er der også større chance for at du bliver ved med at følge den i lang tid.
\item  Det er svært at blive stærkere, hvis man hele tiden hopper fra plan til plan.
\end{tList}



\begin{tList}{Minimum effective dose aka keep it simple}
\item Du skal træne så lidt som muligt, så længe du bliver stærkere af det. Hvorfor det?

\item Du bruger mindre energi, så har du mere krudt til klatring
\item Du bliver ikke overvældet af tusind forskellige øvelser
\item Det er sjovere at klatre end det er at træne. Mindre tid træning, mere tid klatring

\item Væk fra no pain no gain mentalitet. Du er ikke bodybuilder.
\item Hvis din træning er simpel og kort, er der større sandsynlighed for at du gør det.
\end{tList}


\begin{tList}{RIR - I morgen er der også en dag}

\item Lad være med at gå helt til grænsen hver gang. Du bliver ikke stærkere af det, men du bliver trættere.

\item Specifikt for den enkelte øvelse, altså hver sæt i styrketræning skal du ikke give max gas.
\item Der er en masse forskning der indikerer, at det helt klart er muligt at blive stærkere, også selvom man ikke presser sig selv til grænsen hvert sæt. Det er især tilfældet for folk, der ikke har trænet den specifikke øvelse tidligere.
\item Eksempel - Du kan max tage 10 armbøjninger, så tag 8.

\item Lad være med at køre dig selv helt i smadder hver gang du klatrer. Du skal klatre igen senere på ugen. Så høj kvalitet som muligt hver session. Klatring er en teknisk sport.

\item Generelt for hele sessionen. Det giver mening at klatre oftere med kortere sessioner i stedet for færre, som er længere.

\item Hvis du kører en maratonsesh er du sikkert alligevel for træt til at få noget ud af det den sidste times tid.
\end{tList}

\begin{tList}{Progressive overload - stille og roligt mere og mere}

\item Kroppen tilpasser sig det, det du udsætter den for. Derfor skal du løbende udfordre den, ved at tage lidt mere, end du tidligere har gjort.

\item Hvis du ikke presser kroppen, flytter den sig ikke.
\item Hvis du presser den for meget, går den i stykker.

\item Det progressive betyder, at det sker over tid. I skal altså ikke bare smække en masse belastning på systemet og så håbe på det bedste.
\end{tList}

\begin{tList}{Autoregulering - du skal mærke efter og tilpasse dig}

\item Jeg kan ikke være der til hvert set, og fortælle dig, hvor meget du skal løfte. Du skal være klar til at tilpasse vægten, hvis du føler dig svagere eller stærkere en dag. Lad være med at køre på autopilot.

\item Det hjælper dig med at sikre dig, at din krop bliver belastet på den rigtige måde. Ikke for meget, ikke for lidt.
\end{tList}

\begin{tList}{Langsigtet perspektiv}

\item Det handler ikke om at blive stærk nok til at lukke det næste projekt, men at blive stærk år efter år.
\item Start der hvor du rent faktisk er, og ikke der, hvor du ville ønske du var.

\item Ønsketænkning og selvbedrag har forvoldt mange skader.
\end{tList}

\begin{tList}{Hvor kommer gainz fra?}

\item Spis din mad. Sov din søvn. Spis NOK mad. Sov NOK søvn.
\item Relativ styrke vs. Absolut styrke - du behøver nok ikke tabe dig.
\end{tList}

\begin{tList}{Skriv det ned!}

\item Hvis man tager noter til sin træning, er man aldrig i tvivl om, hvorvidt det virker eller ej. Bare kig på tallene.
\item Det bliver meget lettere at vurdere, hvor meget man skal løfte fra gang til gang, når man ikke bare gætter ud fra mavefornemmelse.
\end{tList}

\begin{tList}{Pas på}
\item Sener styrkes markant langsommere end muskler.

\item Hvis du begynder at få ondt i led eller sener, og smerten ikke går i sig selv dagen efter, så stop. Du må gerne være lidt øm i leddene lige bagefter, men det skal gå i sig selv igen.

\item Hvis du tropper op til klatring og tænker "wow, jeg kan virkelig mærke at jeg hangboardede sidste gang", så skal du gå mindre hårdt til den. Der skal ikke bygge sig ømhed op fra uge til uge.
\end{tList}

Du kan hurtigt mærke at du er blevet stærkere. Fedt! Du kan holde fast i ting du ikke kunne før, du kan hænge med mere vægt, mindre kanter, længere tid osv. Tag den med ro. Det kan være berusende at forbedre sig, og det er let at blive grådig. Man får lyst til at give den endnu mere gas, og før man har set sig om, har man revet et eller andet i stykker. Please don't. Følg planen.



Overhængstræning - Hop på overhang som det første i din klatring efter grundig opvarmning 1 gang om ugen i 30 min. Så overhængende som muligt. Du skal kunne lave bevægelser, men det skal ikke være boulders du kan flashe semi-let. Hvis du flasher, skal det være lige på et hængende hår. Det skal altså heller ikke være meget svære projekter, hvor du kun kan lave et enkelt move.

Husk at holde mindst 2 min pause mellem hvert go.

Kilterboardet er godt til det her, kaosvæggen er god til det her. Moonboard er godt, hvis i er stærke nok. Boulderen nedenunder er fin, hvis der er nok boulders, der passer til det niveau i har. Det er der helt klart ikke altid.

Husk progressive overload her og minimum effective dose.

Mest basic hangboard To gange om ugen, før du klatrer, 2 max hangs i 12 sec. Bliv stærkere. Fem sekunders overskud.

Mest basic pull up - 1 gang om ugen, 3 set 5 reps. To reps overskud.

Pull up avanceret - 1 gang om ugen. 4 uger 6 reps 3 set, DELOAD 4 uger 4 reps 3 set. Start forfra. 2 reps overskud første måned, 1 rep overskud anden måned.

3 min rest mellem set.

\begin{tList}{Deload}

\item Hver fjerde uge. Deload betyder bare ingen hang board, ingen pull ups.

\item Hvorfor?
\item Fordi din krop har brug for hvile, også selvom dit sind synes du skal give den gas.

\item Det er tydeligt bevist at deload hjælper kroppen med at hvile, og derfor reducerer skadesrisiko. Der er samtidig også en chance for at det hjælper dig med at blive stærkere i det lange løb, så win-win.

\item "Jamen jeg trænede faktisk ikke sidste torsdag, så skal jeg ikke bare skippe deload den her gang?"
-

\end{tList}

Bare følg planen please…



\end{multicols}
